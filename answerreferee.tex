\documentclass{article}
\usepackage{graphicx} % Required for inserting images



\begin{document}

We would like to thank the anonymous referee for the comments. Every changes in the article were marked by using boldface. \\

\textbf{Major comments:}\\

Q: Given that this paper presents an analysis of a polarimetric dataset to try to find periodic
behaviors in the data, I feel that the introduction section of the paper would benefit from a more
extended discussion of the periodicities found in Betelgeuse that currently exists in the
literature, including uncertainties on those numbers. Currently some of this information is
scattered throughout the paper and not included in the introduction, but it is important for the
reader to have a thorough understanding of what the current state of our understanding of its
periodicities are so that they can put the results of this paper in context.\\

A: We added extra information in the introduction about the different periods found in the literature and their uncertainties. Especially, we did mention the uncertainties found by Jadlovsky et al.2023 and Joyce et al.2020. We also added another reference to Montargès et al.2016 and Chiavassa et al. 2022.\\

Q: I feel that section 2 of the paper would benefit from more discussion on the details of the
dataset. I know that the data have been previously published in Aurière et al. (2016), Mathias et
al. (2018), and López Ariste et al. (2022) and each of those papers includes both a table of the
data they used and the details of the observations, but I think a brief summary of those details in
this paper would help it better stand alone for readers attempting to understand the details of
this work. For example, I think it would be helpful to include information like the total number of
observations used (84 observations are in all three papers, but the paper does not state exactly
what data was cut out of this analysis so that only data from before the great dimming was
used), and how they are divided up between Narval and Neo-Narval (28 of the 84 observations
are from Neo-Narval, but again, it s not easily clear how much of that was not included).
Additionally, I think including a date/timeframe for when the switch between Narval and
Neo-Narval happened is important as changes in instrumentation are important. I also think this
is an appropriate place to discuss how often the data were taken on average, as this information
affects the periods that this dataset would be sensitive to, as well as including the range of dates
over which this dataset was taken (for example, saying something like the data were taken
between November 2013 and December 2019 ). In general, I feel that saying Betelgeuse has
been observed for the last 10 years is vague and if someone is reading this a few years in the
future, they may easily mistake the time period over which the data was taken.\\

A: We added the total number of observations that we used to compute the periodograms. There is a total of 56 observations, we precised that 43\% of the observations were taken between September and December while the other 57\% are taken between January and April. Only Narval data were used in this study, since Neo-Narval was operational after the Great Dimming of Betelgeuse.\\ 

Q: Why was a Lomb-Scargle periodogram not computed for the position angle of the polarization
and shown? I understand that this information is really already tied into the Q and plots, but
so is the \% polarization. In general, I feel that it s best to include \% polarization and position angle
in papers using polarimetric data because that s easier for people not used to working with
Stokes vectors to understand and the combination of the two is the equivalent of showing all of the polarimetric data (just showing \% polarization doesn’t give the full polarimetric details of the
Stokes parameters). If the reason it wasn’t included is because there were no periods found in
the position angle, then I think including it is actually even more important. Section 4, paragraph
2 of the paper states, Once again, we observe the 330 d periodicity to be more prominent in
Stokes than in Q, what may unravel a random situation due to convective motions in the last
years. Convective cells could arise in peculiar positions, where Stokes profile remains
unchanged for month but where Stokes Q could change on shorter timescales. I believe a
periodogram of position angle might help support this statement.\\

A: \\

Q: How was the relative scaling between each 1.8 km/s step maintained in the periodograms to
produce figures 1 through 4?\\

A: For a given periodogram, we computed the highest power of each velocity bin. Then, every other power was compared to this highest value. The closer the power is to the highest power, the redder it is.  \\

Q: In several places, starting in Section 3 and going through the end of the paper, the manuscript
mentions the window function of the Lomb-Scargle periodograms. I think there needs to be a
brief explanation on what the window function is and how that can be used in the interpretation
of the results because only people who are experts in Lomb-Scargle analysis are likely know
what this is.\\

A: We added the following sentence to be sure that readers that are not experts in LS analysis understand the window function "The spectral window reflects the pattern caused by the structure of gaps in the time string. The peaks pointed out by this window cannot be interpreted as ones related to the star."\\

Q: In section 3.2, a Lomb-Scargle periodogram is computed from the photo-center displacements
from polarimetric reconstructed imaging that was presented in López Ariste et al. (2022). The
details of how the photo-center displacements are calculated need to be stated.\\

A: The reference of how the photo-center was computed was added, it was computed using the equations of Chiavassa et al 2022.\\

Q: At first glance, Section 3.3 seems a bit redundant because the paper already is doing a
Lomb-Scargle analysis of the Stokes I parameter. Because Stokes I is total light, it is the light
curve of Betelgeuse. But because the AAVSO light curve and the Stokes I data should be
identical, I think a comparison between the two is important because it could validate the results
of the paper. This section of the paper states that the 400 d and 200 d periods are only
recovered when the full AAVSO light curve is used, but when the AAVSO light curve is reduced
to just the nights where TBL data were obtained we fail to retrieve the periods mentioned in the
literature . However, the Stokes I periodogram does recover the 400 d and 200 d periods. There
is very little discussion about what may cause this difference. Given that they are both
periodograms derived from the total light of the system on the same days, I would have
expected the periodogram of the AAVSO light curve that was trimmed to just include
observations on days the TBL data were taken to match the results from the Stokes I
periodogram. The fact that they do not produce the same periodicities deserves more
discussion than is currently present in the paper. What could be causing this? What does that
mean for the interpretation of the polarimetric results? The paper then goes on to say We can
anticipate that extended polarimetric observations of Betelgeuse will make these periodicities
seen in linear polarization. Due to convective structures, these periodicities will be clearer and
clearer. I don t understand these two sentences because the 200 d and 400 d periods are
already seen in the polarimetric data (there is a ~330 day periodicity in Stokes and
\% polarization, and what appears to be a ~200 day period in Stokes Q and \% polarization). More
observations will likely make their detection levels stronger, but that is unlikely to fix the reason
the Stokes I periodogram isn’t currently reproducing the AAVSO light curve periodogram.\\

A: We computed the periodogram where AAVSO observations correspond to the TBL observation dates. We sum up it in figure 7, where the upper panel represents the average periodogram of Stokes I, the middle panel the periodogram of the AAVSO since 1990 and the lower panel is the periodogram computed from AAVSO observations that correspond to TBL observation dates. We see that the same periods are present in the periodogram from Stokes I and the one where the AAVSO matches TBL observations. Hence, this confirms our results. We mentioned that in both cases, the 400 d period is close to a peak of the window function while the 200 d period is not. \\


\textbf{Minor comments:}\\

Q: Section 1:\\
In the last paragraph of section 1, Lomb-Scargle is misspelled the second time it is used. (The
misspelling is in the sentence that starts In section 3, we seek )\\

A: Corrected.\\

Q: In the last paragraph of section 1 of the paper (line 99 of the manuscript in the reviewer’s copy), the abbreviation LSD is used before the reader is told what it stands for (currently that s in the first paragraph of section 2, lines 122 and 123). I suggest flipping these so what LSD stands for
is clarified the first time it is mentioned.\\

A : We flipped the two sentences.\\

Q: Section 2:\\
In the first sentence of section 2, the acute accent over the first e in the word Télescope is missing.\\

A: Corrected.\\

Q: From López Ariste et al. (2022), I understand that the upgrade between Narval and Neo-Narval
happened during Betelgeuse s great dimming. Therefore, was any neo-Narval data actually
included in the Lomb-Scargle analysis shown? If not, then maybe Neo-Narval shouldn t even be
mentioned in section 2 of the paper.\\

A: We only used Narval data in our paper, it was clarified in the text, at the beginning of Section 3. \\

Q: The total number of lines that go into the LSD profile of each observation should be mentioned,
including the details of how lines are chosen to be included. (e.g. the details of the mask, cut off
for the central depth of the lines compared to the continuum for them to be included, etc.)\\

A: We added information about the mask to produce the LSD profile, those information can also be found in Mathias et al 2018 and Aurière et al 2016.\\

Q: Section 3:\\
How was the Lomb-Scargle analysis performed? Was it using the Lomb-Scargle class in astropy
or some other fitting software (such as Period04)?\\

A: We used the Lomb-Scargle class from Astropy and added a reference in the text.\\ 

Q: "Sometimes, signals at velocities greater than +40 km/s can be found..." is mentioned in the first paragraph of section 3. How much greater than +40 km/s are they? Since the explanation for
signals greater than this velocity is convective processes, is +60 km/s enough to still include
them? If so, maybe just stating that the +60 km/s cut off was used to ensure you always include
the occasional +40 km/s signal is appropriate.\\

A: The peaks in the linear polarization signal do not exceed +50km/s, so the cut off at +60km/s is to add the occasional signal above +40km/s. We precised it in the paper. \\

Q: In the second paragraph of Section 3.1, the paper states that the upper panel depicts the
Lomb-Scargle periodogram of each velocity bin (black line)... , but I believe that this should be
grey lines.\\

A: Thank you for the precision. \\

Q: Figures 1 through 4 all have HRV listed on the y-axis of the bottom plots, but this abbreviation is not defined anywhere. Perhaps it could be stated that it means heliocentric radial velocity in the
caption of figure 1. Additionally, all four plots have the individual periodograms in the top panels,
but there is no y-axis label. Is that axis power on those panels? If so, I believe it should be labeled. Similarly, the color axis in the bottom panels is not explained. (e.g. Red shows a higher power than dark blue).\\

A: The abbreviation of HRV was included in the legend of Figure 1, as well as the signification of the color axis. The label of the y-axis (power) was added in the figures.\\

Q: Line 196 (Section 3.1 paragraph 4), states "...it appears that the primary period is approximately 0.003 d\^-1 (equivalent to 330 d). I believe it should read ...it appears that the primary frequency is approximately 0.003 d\^-1 (equivalent to a 330 d period)." Section 4, paragraph 2 of the paper states, "Once again, we observe the 330 d periodicity to be more prominent in Stokes than in Q, what may unravel a random situation due to convective motions in the last years. Convective cells could arise in peculiar positions, where Stokes profile remains unchanged for month but where Stokes Q could change on shorter timescales." I think the authors mean Once again, we observe the 330 d periodicity to be more prominent in Stokes than in Q, which may unravel a random situation due to convective motions in the last years. Convective cells could arise in peculiar positions, where Stokes profile remains unchanged for months but where Stokes Q could change on shorter timescales.\\

A: We corrected the sentences. \\

Q: Why is a confidence level given for only one periodogram? How was it determined? In my
understanding of Lomb-Scargle analysis, confidence levels aren’t very reliable.\\

A: We computed the confidence level by computing the standard deviation of the mean value of each 100 periodogram. Since the images that we produce from polarimetry imaging are ambiguous, we performed a statistical analysis to see that the variability found in the photo-center displacement was due to statistical reasons, up to 1 sigma (not 2 as mentioned in the text). Concerning the other periodograms, we computed the LS for each velocity bin and we wanted to see if a variability shows up or not. But there is no reason to compute the confidence level in the observations since we are not performing a statistical analysis on the observations. 


\end{document}
